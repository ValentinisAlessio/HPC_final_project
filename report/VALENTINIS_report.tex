\documentclass{article}
\usepackage[utf8]{inputenc}
\usepackage{amsmath,amssymb}
\usepackage{amsfonts}
\usepackage{paralist}
\usepackage{color}
\usepackage[table]{xcolor}
\usepackage{graphicx}
\usepackage{pgfplots}
\usepackage{authblk}
\usepackage{url}
\usepackage{multirow}
\usepackage{booktabs}
\usepackage{blindtext}
\usepackage{adjustbox}
\usepackage{subcaption}

\usepackage{float}
\usepackage[font=small, labelfont=bf, textfont=it, format=hang]{caption}
\usepackage[colorlinks=true, allcolors=blue]{hyperref}
\urlstyle{same}
\usepackage[english,nameinlink]{cleveref}
\usepackage{natbib} 
\crefname{equation}{}{}

\title{HPC Final Project}
\author{Valentinis Alessio [SM3800008]}
\date{27/02/2024}

\begin{document}
	\maketitle
	\tableofcontents
	
	\part{Exercise 1}
	
	The goal of the exercise is to estimate the latency of default openMPI implementation of two collective blocking algorithms, one of which being \textit{broadcast}, varying the virtual topology with which the algorithms operate, along with the map by whcich the cores are allocated.
	My choice resided into:
	\begin{itemize}
		\item \textbf{Broadcast}: varying virtual topologies among \textit{default}, \textit{basic linear}, \textit{flat chain} and \textit{binary tree}.\\
		\item \textbf{Barrier}: varying virtual topologies among \textit{default}, \textit{basic linear}, \textit{double ring} and \textit{bruck}.
	\end{itemize}
	
	The benchmark library used is OSU microbenchmark, available \href{https://mvapich.cse.ohio-state.edu/benchmarks/}{here}.
	The tests were conducted on the EPYC nodes in the ORFEO clusters and all tests were performed following a \textit{bash} script to automatize the process of data gathering.
	
	\section{Broadcast operation}
	
	MPI\_Broadcast is a fundamental collective communication operation provided by the Message Passing Interface (MPI) library. It allows one process, typically referred to as the root process, to efficiently broadcast data to all other processes in a communicator. This operation is crucial for distributing information across multiple processes in parallel computing environments. It's a widely used one-to-all operation that, by default is implemented in multiple ways, in order to optimize the process of sending data to multiple processes.
	We will go into some further details only of those operations which are covered in this study.
	\begin{itemize}
		\item Default: this kind of implementation should authomatically choose the optimal way of sending the information, based on both size of the message and number of cores.\\
		\item Basic linear: it's the way we could think to implement the broadcast algorithm, i.e. the root process sends sequentially to all processes the message, one by one.\\
		\item Chain: In this implementation, as the name suggests, the message is sent from one process to the other, so process0 sends its message to process1, process1 to process2, and so on and so forth.\\
		\item Binary tree: as the name suggests, this virtual topology of the cores is such that they are places in a binary tree fashion, so process0 is in charge of sending the message to process1 and process2, process1 to process3 and process4, and so on.
	\end{itemize}
	
	% INSERT IMAGE OF VIRTUAL TOPOLOGIES
	
	For some of these operations message segmentation is inabled, so the message is divided in chunks, and these chunks are sequentially sent to the other processes. In particular we are referring to binary tree and chain.
	
	\subsection{Data collecting process}
	The process of collecting data has been authomatized through the use of a bash script, available in the GitHub \href{https://github.com/ValentinisAlessio/HPC_final_project}{repository}. 
	To have a better view of the what happens with different mappings of the cores, the same gathering process has been brougth on varying the \textit{--map-by} parameter, and to collect as many data in one shot, the maximum message has been truncated to 2024 bytes, which is anyway well beyond the latency-dominated size region.
	To allow for a better data gathering process, the different mapping of the cores (by core, by socket and by node), the different scripts have been separated.
	
	As previously announced, the tests have been brought up into the EPYC nodes, selecting and prioritizing the acces to two whole nodes. So, due to the characteristics of the computing architecture, I have conduced the test onto a maximum of 256 cores, distributed in four sockets and two nodes.
	
	To have better quality of the data collected, I chose to set 100 warmup operation and 10000 effective operations, from which the average has been taken and recorded.
	
	\subsection{Latency analysis}
	
	The data collection process generated a pretty substantial dataset, providing different combinations for algorithm, cores allocation and message size to analyze the average latency of the communication.
	To tackle this complexity and various combinations, my effort was to sequentially fix some degrees of freedom, while varying the others, one or more at the time. At the end I tried to develop a model that could explain the interaction between the various covariates and the Average Latency measurements.
	
	\subsubsection{Fix algorithm and vary allocation}
	
	In this part we explore the behavior of the various algorithm, for data size of 1 MPI\_CHAR, so 1 byte, by varying the number of cores and cores allocation. This choose of the message size isolates and examines the "pure" latency of the communication.
	
	% INSERIRE IMMAGINE DI ALGORITMI PER 1 CHAR communications
	% 1a default
	% 1b linear
	% 1c chain
	% 1d binary
	
	Concerning the first plot, default algorithm behaves as expected, as we can clearly identify three regions, above all in core and socket mappings, with two noticable jumps and changing in behavior, while node allocation doesn't show remarkable and noticable jumps. The two jumps happen almost where expected, so at 64 cores, when we change socket, and at 128 cores, when we completely change node. It's worth to notice that the jump at the change of socket is by far more accentuated in the core mapping, as in that scenario cores are allocated as close as possible in the same socket. 
	
	Passing at the second plot, regarding the basic linear algorithm, we can observe almost the same behaviour at least for the core and node mapping. In socket mapping we can see that the change of behavior doesn't happen exactly at 128 cores as expexted, suggesting some more complicated model beyond "naive" node/socket changes.
	
	Going to the chain algorithm some similar observations to the previous ones hold, but with some peculiarities.
	In fact, differently from the previous ones, we can see the "serial" nature of the comunication, as the different positioning of the processes deeply impact on the latency measurements, with the node allocation by far diverging from the other two, at least in the one-node situation.
	
	Lastly, going to the binary tree algorithm, the lower latency measurements underscore and emphasize its (expected) superiority compared to the other algorithms. Moreover, we can see that allocating processes by node, we obtain some more stable latency when surpassing the 128 cores, showing some kind of optimized communication when dealing with processes inside the same node.
	Allocating by core or socket, instead, shows some jumps followed by immediate decreases in latency corresponding to the "region change" thresholds, which explanation may reside inside the hierarchical nature of the algorithm. In fact these phenomena could be due to the transition between two different levels of the tree, that may introduce some delay in the communication. Once this transition is passed, the latency decreases, as the subsequent level of the tree gets filled.
	
	\subsubsection{Compare algorithms fixing allocation}
	
	In this second part of our analysis, we fixed the allocation type to \textit{core}, as it shows all the expected changes in behavior of the algorithm, and is the less influenced by eventual iner-node traffic with other communication. Fixing this degree of freedom, we chose once more to analyze the 1 byte scenario, to isolate the pure latency of the communication, and we vary number of cores and type of algorithm.
	
	An initial analysis is provided comparing linear and chain algorithms, as they provide a more understandable background of what's going on under the hood and have a more straight-forward explanation.
	
	% INSERT IMAGE OF ALL MAP BY CORE
	
	The figure above offer some insights into how the algorithm influences across three main regions:
	
	\begin{itemize}
		\item \textbf{Within 64 cores}: in this intra-socket region we can observe almost identical performances, which can be indicative of some optimal speed of communication occuring between cores in the same socket. The slight advantage observed with respect to the default and binary tree algorithm may be due to the fact that the first selects the send policy authomatically, while the latter organizes in a more optimal way the hierarchy of communication.\\
		\item \textbf{From 64 to 128 cores}: in this intra-node region we can begin to observe some more consistent advantage in terms of latency of the two algorithm mentioned above, which suggests that in this situation a choice among implementation may be done.
		\item \textbf{Above 128 core}: in this region we enter inter-node communication, and we can clearly state that the performance of the linear algorithm gets worse and worse, suggesting that with more processes to manage, the binary tree algorithm gains more than something in terms of performance, as the communication is done by a hierarchy processes, and not by one to all.
	\end{itemize}

	\subsubsection{Fixing allocation and vary message size}
	
	In this section we plan to analyze the various performance models among different message size.
	
	% INSERT IMAGES ABOUT DIFFERENT MESSAGE SIZES
	
	Varying message sizes we can observe more peculiar aspects of each implementations: while with a 1 byte message, the difference between implementations is pretty negligible as long we remain into the same node, varying message sizes, we can clearly see how the linear algorithm performs wprse and worse increasing the message size.
	This behavior is due to two main factors: firstly the communication between processes is put as responsibility of the root process, which one by one sends the message to the other processes, while in all other algorithms the communication is organized in a more or less hierarchical way; secondly the linear approach is the only one among the analyzed ones that doesn't imply message segmentation, so in the linear algorithm the message is sent entirely from the root process to the remaining ones. The other methods, on the other hand, aside from the hierarchy of communications, allow for message segmentation, i.e. the message is divided in chunks and sent one chunk at the time via buffer-send (MPI\_ISend), as mentioned in the article (\textit{Here put link to article in the github repo of HPC}).
	
	\subsection{Broadcast performance model}
	As a conclusive step of the analysis I attempted to formulate a performance model to try to better understand the not-so-straight-forward dynamics playing into these algorithms. 
	At firt I gave a try at estimating both latency and bandwidth within point-to-point communications employing again the OSU-microbenchmark library. The intention was to try and develop the Hockney model I found explained in some articles. However, the results obtained didn't fit well with the collected data, suggesting me to try something different. Maybe biased by my mathematical backgroud, or by some freshly sustained exams, I opted for the implementation of a linear model.
	
	% INSERISCI PARTE RIGUARDO MODELLO LINEARE
	
	
	\section{Barrier operation}
	 
\end{document}